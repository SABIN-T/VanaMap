\documentclass[11pt, a4paper]{article}

% Packages for formatting and math
\usepackage[utf8]{inputenc}
\usepackage{geometry}
\usepackage{amsmath, amssymb, amsfonts}
\usepackage{graphicx}
\usepackage{booktabs}
\usepackage{hyperref}
\usepackage{fancyhdr}
\usepackage{float}
\usepackage{algorithm}
\usepackage{algpseudocode}
\usepackage{siunitx}
\usepackage{xcolor}

% Page Setup
\geometry{margin=1in}
\pagestyle{fancy}
\fancyhf{}
\lhead{\textbf{VanaMap Project}}
\rhead{Biometric Simulation Report}
\cfoot{\thepage}

% Title Information
\title{\textbf{VanaMap: A Stochastic Monte Carlo Simulation Model for Indoor Ecosystem Oxygenation and Biometric Aptness Checking}}
\author{
    \textbf{VanaMap Research Team} \\
    \textit{Department of Advanced Agentic Coding} \\
    Google DeepMind / VanaMap Inc.
}
\date{\today}

\begin{document}

\maketitle

\begin{abstract}
    In an era of increasing urbanization and indoor air quality concerns, the integration of distinct flora into living spaces has become a critical public health strategy. This paper presents the mathematical framework and computational logic behind the \textbf{VanaMap Simulation Engine}. We propose a hybrid deterministic-stochastic model that estimates plant oxygen production ($O_2$) rates under varying environmental conditions (temperature, humidity, and luminosity). Utilizing a Monte Carlo method with $N=1000$ iterations per estimation, the model accounts for natural variance in photosynthetic efficiency. Furthermore, we define a \textit{Biometric Aptness Algorithm} that calculates a survival probability score for specific plant species based on localized geospatial weather data.
\end{abstract}

\tableofcontents
\newpage

\section{Introduction}

The physiological interaction between indoor flora and the immediate atmospheric environment is governed by complex biochemical processes, primarily photosynthesis and cellular respiration. While general metrics for plant oxygen output exist (e.g., NASA Clean Air Study, 1989), they often assume static, ideal laboratory conditions. Real-world residential environments, however, fluctuate significantly in temperature, humidity, and light availability.

VanaMap addresses this gap by creating a dynamic simulation environment. Our objective is to predict two key outcome variables:
\begin{enumerate}
    \item \textbf{Net Daily Oxygen Production ($P_{net}$):} The surplus oxygen generated by a specific plant species after accounting for its own respiratory consumption.
    \item \textbf{Biometric Aptness Score ($S_{apt}$):} A normalized index representing the likelihood of a plant thriving in a user's specific geolocation.
\end{enumerate}

This document details the mathematical equations, algorithmic logic, and computational implementation used in the VanaMap application.

\section{Mathematical Framework}

\subsection{Photosynthesis and Oxygen Kinetics}
The foundational chemical equation for photosynthesis acts as the baseline for our deterministic model:
\begin{equation}
    6CO_2 + 6H_2O + \text{photons} \rightarrow C_6H_{12}O_6 + 6O_2
\end{equation}
From stoichiometry, 1 mole of carbon dioxide fixed yields 1 mole of oxygen gas. At Standard Temperature and Pressure (STP), 1 mole of gas occupies $22.4$ liters.

We define the \textit{Base Photosynthetic Rate} ($R_{base}$) in units of $\mu mol(CO_2) \cdot m^{-2} \cdot s^{-1}$. Typical values for our database entry $i$ ($R_{base, i}$) range from 10 to 25 depending on the species (e.g., \textit{Sansevieria trifasciata} vs. \textit{Dracaena}).

\subsection{Environmental Modulation Factors}
The base rate is rarely achieved in domestic settings. We introduce modulation coefficients $\eta \in [0, 1]$ to scale performance.

\subsubsection{Temperature Coefficient ($\eta_{temp}$)}
Enzymatic activity in the Calvin cycle follows a Gaussian distribution centered around an optimal temperature $T_{opt}$ (typically \SI{25}{\celsius}). We model this as:
\begin{equation}
    \eta_{temp}(T) = \exp\left( - \frac{(T - T_{opt})^2}{2\sigma^2} \right)
\end{equation}
Where:
\begin{itemize}
    \item $T$ is the current ambient temperature.
    \item $T_{opt} = 25$ for tropical houseplants.
    \item $\sigma = 10$, defining the tolerance width.
\end{itemize}
This function yields $\eta_{temp} \approx 1.0$ at \SI{25}{\celsius} and decays to $\approx 0.1$ at \SI{5}{\celsius} or \SI{45}{\celsius}, simulating heat/cold stress.

\subsubsection{Humidity Coefficient ($\eta_{hum}$)}
Stomatal conductance is regulated by vapor pressure deficit. We model this as a piecewise linear function based on relative humidity ($H$):
\begin{equation}
    \eta_{hum}(H) = 
    \begin{cases} 
      0.7 + 0.3 \left(\frac{H}{30}\right) & 0 \le H < 30 \\
      1.0 & 30 \le H \le 80 \\
      1.0 - 0.15 \left(\frac{H-80}{20}\right) & 80 < H \le 100 
   \end{cases}
\end{equation}
Below 30\% humidity, stomata close to conserve water, reducing $CO_2$ intake. Above 80\%, transpiration efficiency drops, risking fungal pathogens.

\subsubsection{Luminosity Cycle ($\eta_{light}$)}
Light intensity $I(t)$ follows a diurnal sine wave pattern during photoperiods ($t_{rise} < t < t_{set}$):
\begin{equation}
    \eta_{light}(t) = \max\left(0, \sin\left(\frac{t - t_{rise}}{t_{set} - t_{rise}} \pi \right)\right)
\end{equation}
Assuming a 12-hour cycle (06:00 to 18:00):
\begin{equation}
    \eta_{light}(t) = \sin\left(\frac{t - 6}{12}\pi\right), \quad 6 \le t \le 18
\end{equation}
For $t < 6$ or $t > 18$, $\eta_{light} = 0$, representing nighttime where photosynthesis halts and only respiration occurs.

\newpage

\section{Stochastic Monte Carlo Simulation}

To account for biological variance, cloud cover, and sensor error in VanaMap, we do not rely on a single deterministic calculation. Instead, we implement a Monte Carlo simulation.

\subsection{Random Variables}
We introduce three stochastic perturbation factors:
\begin{itemize}
    \item $\epsilon_{bio} \sim \mathcal{N}(1, 0.15^2)$: Biological variance (genetics, health). Standard deviation of 15\%.
    \item $\epsilon_{cloud} \in \{0.7, 0.85, 1.0\}$: Discrete distribution reflecting weather (Overcast, Partly Cloudy, Clear).
    \item $\epsilon_{noise} \sim \mathcal{U}(0.95, 1.05)$: Uniform sensor noise.
\end{itemize}

\subsection{Simulation Algorithm}
For a given plant $i$, the simulator executes $N=1000$ iterations.

\begin{algorithm}[H]
\caption{Oxygen Production Simulation}
\begin{algorithmic}[1]
\State \textbf{Input:} Plant Type $P$, Temperature $T$, Humidity $H$
\State \textbf{Initialize:} $TotalO_2 \leftarrow 0$
\For{$k \leftarrow 1$ to $N$}
    \State Sample $\epsilon_{bio} \sim \mathcal{N}(1, 0.15)$
    \State Sample $\epsilon_{cloud}$ based on weather probabilities
    \State Calculate $\eta_{temp}(T)$ and $\eta_{hum}(H)$
    \State Integration over daylight hours:
    \State $DailyOutput_k \leftarrow R_{base, P} \times \eta_{temp} \times \eta_{hum} \times \epsilon_{bio} \times \int_{6}^{18} (\eta_{light}(t) \times \epsilon_{cloud}) \, dt$
    \State Convert $\mu mol \to Liters$ using factor $C_{conv}$
    \State $TotalO_2 \leftarrow TotalO_2 + DailyOutput_k$
\EndFor
\State $MeanO_2 \leftarrow TotalO_2 / N$
\State Calculate Respiration Loss ($Loss \approx 0.1 \times MeanO_2$ assuming $Q_{10}$ scaling)
\State \textbf{Return} $NetO_2 = MeanO_2 - Loss$
\end{algorithmic}
\end{algorithm}

\section{Biometric Aptness Scoring}

Beyond oxygen production, VanaMap determines if a plant will survive in a user's location. This is the "Smart Match" feature.

\subsection{Scoring Logic}
We define a score $S_{raw} \in \mathbb{R}$ composed of weighted sub-scores:
\begin{equation}
    S_{raw} = w_T \cdot f_T(\Delta T) + w_H \cdot f_H(\Delta H) + w_R \cdot \mathbb{I}_{resilient} + w_{AQI} \cdot f_{AQI}(Q)
\end{equation}

Where:
\begin{itemize}
    \item $w_T = 40$: Temperature weight (Critical).
    \item $w_H = 30$: Humidity weight (High importance).
    \item $w_R = 20$: Resilience/Hardiness bonus.
    \item $w_{AQI} = 10$: Relevance to local pollution levels.
\end{itemize}

\subsection{Penalty Functions}
Stress penalties are non-linear. For temperature difference $\Delta T$ outside the ideal buffer:
\begin{equation}
    f_T(\Delta T) = \max(0, 1 - \alpha |\Delta T|)
\end{equation}
If the current temperature $T_{curr}$ deviates beyond $\pm 10^\circ C$ of the plant's survival limit, a \textbf{Critical Failure} flag is raised, forcing $S_{apt} = 0$.

\newpage

\section{Results and Validation}

\subsection{Human Consumption Modelling}
To provide actionable insights, we model human oxygen consumption to determine the required flora density.
\begin{equation}
    C_{human} \approx 550 \, L/day
\end{equation}
Based on basal metabolic rate (BMR) and average activity levels. The number of plants required ($N_{plants}$) is derived as:
\begin{equation}
    N_{plants} = \lceil \frac{N_{people} \times 550}{NetO_2} \rceil
\end{equation}

\subsection{Case Study: Sansevieria trifasciata}
Running the simulation for a Snake Plant under standard office conditions ($T=\SI{22}{\celsius}$, $H=45\%$):

\begin{table}[H]
\centering
\caption{Simulation Output for Sansevieria}
\begin{tabular}{lcc}
\toprule
\textbf{Parameter} & \textbf{Value} & \textbf{Unit} \\
\midrule
Base Rate ($R_{base}$) & 25 & $\mu mol \cdot m^{-2} s^{-1}$ \\
Temp Coeff ($\eta_{temp}$) & 0.96 & Unitless \\
Humidity Coeff ($\eta_{hum}$) & 1.0 & Unitless \\
Daylight Integral & 7.64 & Effective Hours \\
\textbf{Mean Gross $O_2$} & \textbf{11.2} & \textbf{Liters/Day} \\
Respiration Cost (CAM) & -1.1 & Liters/Day \\
\textbf{Net $O_2$ Output} & \textbf{10.1} & \textbf{Liters/Day} \\
\bottomrule
\end{tabular}
\end{table}

\subsection{System Diagnostics}
The VanaMap platform continuously monitors the latencies of these calculations.
\begin{figure}[H]
    \centering
    \begin{verbatim}
    +--------------------------------+
    | SYSTEM HEALTH                  |
    +--------------------------------+
    | CPU Usage:        12%          |
    | Sim Latency:      12ms         |
    | Database Sync:    Active       |
    | Neural Net:       Online       |
    +--------------------------------+
    \end{verbatim}
    \caption{Admin Dashboard Diagnostics displaying simulation health.}
\end{figure}

\section{Code Implementation}
The following TypeScript snippet demonstrates the core aggregation logic used in the frontend `logic.ts`:

\begin{verbatim}
// TypeScript Implementation of Scoring
let tempScore = 40;
const buffer = 3;

if (currentTemp < (plant.idealTempMin - buffer)) {
    const diff = (plant.idealTempMin - buffer) - currentTemp;
    tempScore -= (diff * 4); // Severity penalty
} 

// Clamp Score
totalRaw = Math.max(0, Math.min(100, totalRaw));
\end{verbatim}

\section{Conclusion}

The VanaMap biometric simulation bridges the gap between theoretical botany and practical indoor environmental control. By employing a Monte Carlo approach, we account for the inherent unpredictability of biological organisms. The resulting data enables users to make scientifically informed decisions about their indoor ecosystems, optimizing for both plant survival (Aptness) and human health benefit (Oxygenation).

Future work will integrate real-time sensor data (IoT) to replace static weather API inputs, creating a closed-loop feedback system for automated environment adjustment.

\end{document}
